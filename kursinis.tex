\documentclass{VUMIFPSkursinis}
\usepackage{algorithmicx}
\usepackage{algorithm}
\usepackage{algpseudocode}
\usepackage{amsfonts}
\usepackage{amsmath}
\usepackage{bm}
\usepackage{caption}
\usepackage{color}
\usepackage{float}
\usepackage{graphicx}
\usepackage{listings}
\usepackage{subfig}
\usepackage{wrapfig}

% Titulinio aprašas
\university{Vilniaus universitetas}
\faculty{Matematikos ir informatikos fakultetas}
\department{Programų sistemų katedra}
\papertype{Kursinis darbas}
\title{Blokų grandinių duomenų bazių analizė }
\titleineng{Blockchain Database Analysis}
\status{3 kurso 3 grupės studentas}
\author{Justas Tvarijonas}
% \secondauthor{Vardonis Pavardonis}   % Pridėti antrą autorių
\supervisor{dr. Vytautas Valaitis}
\date{Vilnius – \the\year}

% Nustatymai
% \setmainfont{Palemonas}   % Pakeisti teksto šriftą į Palemonas (turi būti įdiegtas sistemoje)
\bibliography{bibliografija}

\begin{document}
\maketitle

\tableofcontents

\sectionnonum{Įvadas}
Per pastaruosius keletą metų blokų grandinių technologija susilaukė didelio žmonių susidomėjimo. 
Šis susidomėjimas daugiausiai kilo dėl išpopulerėjusių kriptovaliutų, tokių kaip Bitcoin, Etherium, Litecoin ir daugeliu kitų 
kurios ir yra paremtos blokų grandinių technologija. Šią technologiją 2008 metais sukūrė Satošis Nakamoto  \cite{BlockChain}. 
2009 metais Nakamoto implementavo blokų grandinių technologiją sukurdamas Bitcoin kriptovaliutą \cite{Bitcoin}. 
Nors, šiuo metu, žmonių susidomėjimas kripto valiutomis ir yra sumažėjęs \cite{Trends}, tačiau informacinių technologijų industrija 
mato daugiau blokų grandinių panaudojimo atveju negu tik kripto valiutos. Vienas iš blokų grandinių panaudojimo atvėjų yra 
blokų grandinių duomenų bazės. Reliacinės ir dokumentų duomenų bazės ilgą laiką buvo pagrindinis duomenų saugojimo būdas. 
Tačiau šios duomenų bazės turi ir savo trūkumų, saugant duomenis tradicinėsė duomenų bazėse kyla duomenų integralumo problemos \cite{Integrity}
. 
Naudojant duomenų bazes financinėms transakcijoms sekti kyla dvigumo pinigų išleidimo problema\cite{Double}
. Naudojantis tradicinėmis duomenų bazėmis 
taip pat kyla pasitikėjimo problema, visa duomenų prieeiga yra trečiosios šalies valdžioje, ir vartotojas turi pasitikėti, kad duomenys nebus pakeisti be jo žinios.
Per pastaruosius kelis metus šias problemas
 buvo stengtasi išspręsti kuriant duomenų bazes paremtas blokų grandinių technologija. Privačios blokų grandinių duomenų bazės užtikrina pasitikėjimą, nes kiekvienas vartotojas turi visą duomenų 
bazės kopiją. Darant pakeitimus tokioje duomenų bazėje kiekvienas vartotojas turi sutikti su daromais pakeitimais. Blokų grandinių duomenų bazės išsprendžia duomenų integralumo ir
dvigumo pinigų išleidimo problemą, nes kiekvienas mazgas blokų grandinėje tinkle gali palyginti savo turimus duomenis su kitais mazgais. Šiuo metu try populiariausios blokų grandinių duomenų 
bazės yra: Corda, BigchainDB ir Hyperleadger Fabric. Nors blokų grandinės išsprendžia autoriaus išvardytas problemas, tačiau finansinėms transakcijos svarbus ir greitis. Šiuo darbu autorius 
sieks palyginti reliacinių duomenų bazių greitį finansinėms transakcijoms skaityti ir įrašyti su trimis aukščiau išvardintomis blokų grandinių duomenų bazėmis. 


\printbibliography[heading=bibintoc]  









\end{document}
