\documentclass{VUMIFPSkursinis}
\usepackage{algorithmicx}
\usepackage{algorithm}
\usepackage{algpseudocode}
\usepackage{amsfonts}
\usepackage{amsmath}
\usepackage{bm}
\usepackage{caption}
\usepackage{color}
\usepackage{float}
\usepackage{graphicx}
\usepackage{listings}
\usepackage{subfig}
\usepackage{wrapfig}

% Titulinio aprašas
\university{Vilniaus universitetas}
\faculty{Matematikos ir informatikos fakultetas}
\department{Programų sistemų katedra}
\papertype{Kursinis darbas}
\title{Blokų grandinių duomenų bazės finansinių duomenų kaupimui}
\titleineng{Blockchain databases for financial data}
\status{3 kurso 3 grupės studentas}
\author{Matas Savickis}
\supervisor{Vytautas Valaitis, Asist., Dr.}
\date{Vilnius – \the\year}
\begin{document}
\maketitle

\tableofcontents


\sectionnonum{Įvadas}
\section{First section}
Per pastaruosius keletą metų blokų grandinių technologija susilaukė didelio žmonių susidomėjimo. 
Šis susidomėjimas daugiausiai kilo dėl išpopulerėjusių kriptovaliutų, tokių kaip Bitcoin, Etherium, Litecoin ir daugeliu kitų 
kurios ir yra paremtos blokų grandinių technologija. Šią technologiją sukūrė 2008 metais sukūrė Satošis Nakamoto% \cite{1}
. 
2009 metais Nakamoto implementavo blokų grandinių technologiją sukurdamas Bitcoin kriptovaliutą %\cite{2}
. 
Nors, šiuo metu, \cite{lampost94}žmonių susidomėjimas kripto valiutomis ir yra sumažėjęs %\cite{3}
, tačiau informacinių technologijų industirja 
mato daugiau blokų grandinių panaudojimo atveju negu tik kripto valiutos. Vienas iš blokų grandinių panaudojimo atvėjų yra 
blokų grandinių duomenų bazės. Reliacinės ir dokumentų duomenų bazės ilgą laiką buvo pagrindinis duomenų saugojimo būdas. 
Tačiau šios duomenų bazės turi ir savo trūkumų, saugant duomenis tradicinėsė duomenų bazėse kyla duomenų integralumo problemos %\cite{4}
. 
Naudojant duomenų bazes financinėms transakcijoms sekti kyla dvigumo pinigų išleidimo problema% \cite{5}
. Naudojantis tradicinėmis duomenų bazėmis 
taip pat kyla pasitikėjimo problema, visa duomenų prieeiga yra trečiosios šalies valdžioje, ir vartotojas turi pasitikėti, kad duomenys nebus pakeisti be jo žinios.
Per pastaruosius kelis metus šias problemas buvo stengtasi išspręsti kuriant duomenų bazes paremtas blokų grandinių technologija.

\begin{thebibliography}{999}

\bibitem{lamport94}
  Leslie Lamport,
  \emph{\LaTeX: A Document Preparation System}.
  Addison Wesley, Massachusetts,
  2nd Edition,
  1994.

\end{thebibliography}


%\begin{thebibliography}{9}
%\bibitem{1} S. Nakamoto. A peer-to-peer electronic cash system. 2008.

%\bibitem{2} Bitcoin: A Peer-to-Peer Electronic Cash System https://bitcoin.org/bitcoin.pdf

%\bibitem{3} https://trends.google.com/trends/explore?date=all&geo=US&q=bitcoin,Etherium,Litecoin

%\bibitem{4} E.B. Fernandez, R.C. Summers, and C. Wood, “Database Security and Integrity,” Addison-Wesley, February 1981. 

%\bibitem{5} Distributed Double Spending Prevention Jaap-Henk Hoepman


%\end{document}
